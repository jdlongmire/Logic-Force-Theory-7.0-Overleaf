\section{Mathematical Formalization of Logic Force Theory}

Logic Force Theory (LFT) introduces logical necessity as a governing principle in quantum mechanics. This formalization extends the standard quantum mechanical framework by incorporating constraints that ensure deterministic state evolution through logical consistency.

\subsection{Core Axioms and Logical Constraints}
LFT is based on three fundamental axioms that constrain quantum state evolution:

\begin{itemize}
    \item \textbf{Axiom 1: Logical Necessity is a Governing Constraint} \\
    All physically realized states must conform to fundamental logical principles (identity, non-contradiction, excluded middle). Any state violating these principles is inherently unphysical.
    
    \item \textbf{Axiom 2: Universal Logic Field (ULF) as a Governing Structure} \\
    The ULF, denoted as $L_{ULF}$, acts as a global operator that ensures logical consistency across all quantum states:
    \[
    L_{ULF}: S \to S_L
    \]
    where $S$ is the space of all possible quantum states, and $S_L$ is the subset of logically permissible states.

    \item \textbf{Axiom 3: Information Precedes Physical Instantiation} \\
    The state of any system can be represented as an informational configuration that follows logical constraints. Logical necessity operates on this informational level before physical instantiation.
\end{itemize}

\subsection{The Logical Constraint Hamiltonian $H_L$}
LFT modifies the Schrödinger equation by introducing the **Logical Constraint Hamiltonian** ($H_L$) to enforce logical consistency in quantum state evolution. The generalized form of the Schrödinger equation becomes:
\[
i\hbar \frac{\partial}{\partial t} |\psi(t)\rangle = (H + H_L) |\psi(t)\rangle
\]
where $H$ is the standard quantum Hamiltonian and $H_L$ represents the logical constraints imposed on quantum states.

\subsubsection{Explicit Form of $H_L$}
We define the Logical Constraint Hamiltonian $H_L$ as:
\[
H_L = P_L H P_L
\]
where $P_L$ is a **projection operator** that ensures the logical consistency of quantum states:
\[
P_L^2 = P_L \quad \text{(idempotent)}
\]
This ensures that only logically permissible states evolve, and that the evolution remains unitary.

\subsubsection{Time-Dependent Logical Filtering}
In more advanced formulations, $H_L$ can be generalized to include time-dependent logical filtering. This introduces a dynamic evolution of logical constraints over time:
\[
H_L = P_L H P_L + \lambda(t) L_{ULF}
\]
where $\lambda(t)$ is a time-dependent coefficient that adjusts the strength of the logical constraint.

\subsection{Logical Filtering and Probabilities}
LFT introduces a logical filtering operator to modify the probability distributions of measurement outcomes. The **Born rule** is modified under LFT to account for logical selection, ensuring that only logically consistent states are realized in measurement. The modified probability distribution $p_L(s)$ for a measurement outcome $s$ is given by:
\[
p_L(s) = \langle s | L_{ULF} | s \rangle \, p(s)
\]
where $p(s)$ is the standard probability from the Born rule, and $L_{ULF}$ is the logical filtering operator. The normalization factor $Z$ ensures that the probabilities sum to 1:
\[
p_L(s) = \frac{\langle s | L_{ULF} | s \rangle \, p(s)}{Z}
\]

\subsection{Logical Entropy Suppression and Information Theory}
LFT introduces a new form of entropy suppression, ensuring that logical necessity reduces uncertainty in a structured manner. The standard Shannon entropy \cite{shannon1948} is modified under LFT as follows:
\[
H(S) = - \sum_i p(s_i) \log_2 p(s_i)
\]
where \( p(s_i) \) is the probability of the state \( s_i \) in the system, and \( H(S) \) represents the Shannon entropy. Shannon's work \cite{shannon1948} introduced this measure as a way to quantify uncertainty in information theory.

LFT imposes an additional constraint through the logical necessity operator \( L_{ULF} \):
\[
H_L(S) = - \sum_i p_L(s_i) \log_2 p_L(s_i)
\]
where \( p_L(s_i) \) is the probability distribution under logical filtering. To ensure entropy follows a deterministic reduction pattern, LFT predicts the following time evolution of logical entropy:
\[
H_L(S,t) = H(S) e^{-\alpha t}
\]
where \( \alpha \) represents the logical constraint enforcement rate, and \( H(S) \) is the standard Shannon entropy.

\subsection{The Role of PR = L(S) in LFT}
The equation \( PR = L(S) \) plays a central role in the application of logical filtering to quantum systems. Here, $PR$ represents the logical probability of a system's state, and $L(S)$ represents the logical filter applied to the system's state space. This relationship ensures that the system evolves in such a way that only logically consistent outcomes are realized.

The logical probability, \( PR \), can be understood as a measure of how likely a given state is to occur under the logical constraints imposed by $L(S)$. This constraint refines the standard probabilistic treatment in quantum mechanics by ensuring that only permissible states, which adhere to logical consistency, are observed.

\subsection{Summary of Mathematical Foundations}
- The Universal Logic Field ($L_{ULF}$) enforces logical constraints on quantum states.
- The Logical Constraint Hamiltonian ($H_L$) modifies quantum evolution by enforcing deterministic state selection.
- Measurement probabilities follow logical selection rather than stochastic randomness.
- Entropy suppression follows a deterministic decay function, reducing uncertainty over time.
- The equation \( PR = L(S) \) encapsulates the core principle of logical filtering in quantum systems.
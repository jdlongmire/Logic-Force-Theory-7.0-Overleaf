\section{Introduction \& Motivation}

Quantum mechanics (QM) has been extraordinarily successful in predicting experimental outcomes, yet it remains incomplete in its interpretational foundation. Several unresolved issues suggest the need for additional theoretical structure.

\subsection{Wavefunction Collapse}
Standard QM treats measurement as a stochastic process governed by the Born rule. The precise mechanism of wavefunction collapse remains ambiguous. This ambiguity has been a long-standing issue in the interpretation of quantum mechanics.

\subsection{Bell Test Violations}
Experimental tests of Bell's inequality consistently show stronger-than-classical correlations in entangled systems. However, these correlations are often explained by nonlocality, with no clear mechanism beyond this interpretation.

\subsection{Measurement Problem}
The Copenhagen interpretation of quantum mechanics implies observer-dependent reality, which conflicts with the principle of an objective external world. This raises critical questions about the role of the observer and the nature of quantum state reduction.

\subsection{Decoherence and Entropy Evolution}
Quantum-to-classical transitions are described by decoherence, yet entropy growth follows stochastic thermodynamic laws rather than deterministic constraints. This presents another puzzle in understanding how classical behavior emerges from quantum systems.

\subsection{Motivation for Logic Force Theory (LFT)}
Logic Force Theory (LFT) proposes that these ambiguities arise due to the absence of a key constraint: logical necessity as a governing principle in quantum mechanics. By introducing a global logical structure, LFT aims to:
\begin{itemize}
    \item Ensure deterministic state selection rather than stochastic collapse.
    \item Explain Bell test results through structured entanglement constraints.
    \item Introduce logical entropy suppression, refining decoherence models.
    \item Provide a unified information-theoretic framework for physics.
\end{itemize}

\subsection{The Universal Logic Field (ULF): A Non-Physical Governing Constraint}
The Universal Logic Field (ULF) is introduced in LFT as a non-physical constraint that governs quantum state evolution. Unlike traditional physical fields such as electromagnetism or gravity, the ULF:
\begin{itemize}
    \item Does not interact via force carriers or local interactions.
    \item Operates at the informational level, ensuring quantum systems adhere to logical constraints.
    \item Serves as a global constraint rather than a localized force.
\end{itemize}
The ULF is conceptually similar to mathematical symmetries in physics, such as those in Noether’s theorem, where conserved quantities arise from symmetry constraints rather than from direct physical forces. Similarly, the ULF ensures that quantum state evolution remains logically self-consistent.

\subsection{Historical Foundations}
The tension between determinism and probability in quantum physics has existed since its inception. 
\begin{itemize}
    \item Albert Einstein’s criticism of QM: “God does not play dice with the universe.” Einstein objected to probabilistic interpretations, seeking an underlying deterministic structure.
    \item John Bell’s theorem (1964): Demonstrated that any hidden-variable theory must be nonlocal. This led to experimental tests of Bell inequalities, confirming quantum entanglement but without a deterministic explanation.
    \item Weak measurement experiments (2011, Kocsis et al.): Showed evidence that quantum trajectories may follow deterministic paths, challenging the purely probabilistic nature of QM.
\end{itemize}
LFT builds upon these foundational insights, proposing that logical constraints preselect allowable state transitions, resolving many of QM’s paradoxes.

\subsection{Scope and Purpose of This Paper}
This paper develops Logic Force Theory as a mathematically rigorous framework that:
\begin{itemize}
    \item Formally defines logical necessity as a constraint in quantum mechanics.
    \item Establishes the Universal Logic Field (ULF) as a governing structure.
    \item Demonstrates LFT’s alignment with existing experimental data.
    \item Presents computational validation simulations supporting LFT’s predictions.
    \item Proposes future experimental tests to differentiate LFT from standard QM.
\end{itemize}
The remainder of this work is structured as follows:
\begin{itemize}
    \item Section 2 establishes the mathematical framework of LFT.
    \item Section 3 presents experimental validation and computational results.
    \item Section 4 explores LFT’s implications in quantum information theory, computing, and classical physics.
    \item Section 5 compares LFT to other interpretations of quantum mechanics.
    \item Section 6 discusses limitations, open questions, and future research directions.
    \item Section 7 presents LFT as a conceptual framework for quantum foundations, addressing fundamental shifts in physics.
    \item Section 8 provides a comprehensive reference list supporting LFT’s theoretical and experimental foundation.
\end{itemize}